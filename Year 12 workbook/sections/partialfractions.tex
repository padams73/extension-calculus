\documentclass[../main.tex]{subfiles}
\graphicspath{{\subfix{../images/}}}
\begin{document}
\section{Partial fractions}
Partial fraction decomposition is the process of splitting a fraction up into a sum/difference of fractions. It is particularly useful with integration and also with telescoping sums.

We use this approach when the numerator has a lower degree (power) than the denominator.

E.g. $\frac{1}{x^2+x}$

The first step is to factorise the denominator.

$\frac{1}{x^2+x}=\frac{1}{x(x+1)}$

Then we create a new fraction for each factor, putting new variables in the numerators.

$\frac{1}{x(x+1)}=\frac{A}{x}+\frac{B}{x+1}$

Now we just need to work out the values of $A$ and $B$.

To do this, we multiply through by the denominator of the original fraction so we no longer have fractions:

$\frac{1}{x(x+1)}=\frac{A}{x}+\frac{B}{x+1} \Rightarrow 1=A(x+1)+Bx$

To find the values of $A$ and $B$, we can just equate the coefficients of the $x$ terms and also the constants.

$x$-terms: $0=A+B$

Constant: $1=A$

Therefore, we know that $A$ must be equal to 1, and since $A + B = 0$, $B=-1$

So, we have our answer:

$\frac{1}{x^2+x}=\frac{1}{x}-\frac{1}{x+1}$

For example,

$\frac{5x-4}{x^2-x-2}$

$=\frac{5x-4}{(x+1)(x-2)}=\frac{A}{x+1}+\frac{B}{x-2}$

$5x-4=A(x-2)+B(x+1)$

$5x-4=Ax-2A+Bx+B$

Equating coefficients and constants:

$x$-terms: $5=A+B$

Constants: $-4=-2A+B$

Solving simultaneously, we get $A=3$ and $B=2$

Giving our answer:

$\frac{5x-4}{x^2-x-2}=\frac{3}{x+1}+\frac{2}{x-2}$

\subsection*{Using critical values}
You can also find A and B by substituting the critical values of each factor into the equation. The critical value is the value for $x$ that would make the bracket equal to zero.

For example, from the example above, substituting the critical values of -1 and 2 gives:

$5x-4=A(x-2)+B(x+1)$

$5(-1)-4=A(-1-2)+0$

$-9=-3A \Rightarrow A=3$

$5(2)-4=0+B(2+1)$

$6=3B \Rightarrow B=2$

Giving the same answer: $\frac{3}{x+1}+\frac{2}{x-2}$

\subsection*{Fractions where one of the denominator factors has a higher power}

When you factorise the denominator and find that one of the factors has a power greater than 1, such as $x^2$, the numerator in the partial fraction will need to be only one degree less. 

In this case, it would be linear, so needs to have the form $Ax + B$.
If the factor was a higher power such as $x^3$, then the numerator would be degree 2, and would be in the form $Ax^2+Bx+c$

For example,

$\frac{1}{x^3+x^2}=\frac{1}{x^2(x+1)}=\frac{Ax+B}{x^2}+\frac{C}{x+1}$

Multiplying everything by $x^2(x-1)$

$1=(Ax+B)(x+1)+Cx^2$

$1=(A+C)x^2+(A+B)x+B$

Equating coefficients and constants:

$x^2$-terms : $A+C=0$

$x$-terms : $A+B=0$

Constant : $B=1$

Solving simultaneously, $A=-1, B=1, C=1$

Giving us the partial fraction $\frac{-x+1}{x^2}+\frac{1}{x+1}$

Another example,

$\frac{2x-1}{x^3+x}=\frac{2x-1}{x(x^2+1)}=\frac{A}{x}+\frac{Bx+C}{x^2+1}$

Multiplying everything by $x(x^2+1)$

$2x-1=A(x^2+1)+x(Bx+C)$

$2x-1=(A+B)x^2+A+Cx$

Equating coefficients and constant:

$x^2$-term : $A+B=0$

$x$-term : $C=2$

Constant : $A=-1$

Solving simultaneously, $A=-1, B=1, C=2$

Giving us the partial fraction: -$\frac{1}{x}+\frac{x+2}{x^2+1}$

\subsection*{Fractions with repeated factors in the denominator}
Sometimes you will get a denominator with a repeated factor, such as $\frac{x+2}{(2x+3)^2}$

In this case, we need a partial fraction for exponent from 1 upwards. Because it is a power of 2, there will be 2 partial fractions:

$\frac{x+2}{(2x+3)^2}=\frac{A}{2x+3}+\frac{B}{(2x+3)^2}$

Multiplying everything by $(2x+3)^2$

$x+2=A(2x+3)+B$

$x+2=2Ax+3A+B$

Equating coefficients and constant:

$x$-term : $2A=1$

Constant : $3A+B=2$

Solving simultaneously, $A=\frac{1}{2}, B=\frac{1}{2}$

Therefore, our partial fractions are $\frac{1}{2(2x+3)}+\frac{1}{2(2x+3)^2}$



\pagebreak

\subsection*{Questions}
(Answers - page \pageref*{Partial fractions answers})
\label{partial fractions}

Convert the fractions into a sum of fractions
\begin{enumerate}[itemsep=1cm]
    \item 
    $\frac{x+5}{(x-3)(x+1)}$

    \item 
    $\frac{x+26}{x^2+3x-10}$

    \item 
    $\frac{4x-8}{x^2-8x+15}$

    \item 
    $\frac{12x-1}{x^2+x-12}$

    \item 
    $\frac{x-5}{(x-2)^2}$

    \item 
    $\frac{5x+4}{(x-1)(x+2)^2}$

    \item 
    $\frac{2x^2-5x+7}{(x-2)(x-1)^2}$

    \item 
    $\frac{6-x}{(1-x)(4+x^2)}$

    \item 
    $\frac{5x+2}{(x+1)(x^2-4)}$
\end{enumerate}
\end{document}