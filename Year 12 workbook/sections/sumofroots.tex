\documentclass[../main.tex]{subfiles}
\graphicspath{{\subfix{../images/}}}
\begin{document}
\section{Sum of roots of polynomials}
The sum of the roots of any polynomial in the form \(ax^n+bx^{n-1}+cx^{n-2}+...+z=0\) will always be equal to \(-\frac{b}{a}\).

We can see that this holds for quadratics in the form \(ax^2+bx+c=0\) as we know from when we factorise we need to find two numbers that multiply to \(c\) and add to \(b\). This gives us the factors, and since the roots are \((x-x_1)\), it means the sum will be \(-b\) (which is \(\frac{-b}{1}\) since \(a=1\) here).

We can also see this from the quadratic equation:
\(x=\frac{-b\pm\sqrt{b^2-4ac}}{2a}\)

If we add the two roots, we get:

\(\frac{-b+\sqrt{b^2-4ac}}{2a}+\frac{-b-\sqrt{b^2-4ac}}{2a}=-\frac{2b}{2a}=-\frac{b}{a}\)

This holds for all polynomials. For example, in the polynomial \(p(x)=2x^4-x^3+2x-1=0\) we know the four roots will sum to \(\frac{1}{2}\), since \(-(-\frac{1}{2})=\frac{1}{2}\).

\pagebreak

\subsection*{Questions}
(Answers - page {\pageref*{Sum of roots answers}})
\label{Sum of roots}
\begin{enumerate}
    \item Find the roots of the equation \(z^{11}=1\). Use this to show that:
    
    \(\cos{(\frac{2\pi}{11})}+\cos{(\frac{4\pi}{11})+\cos{(\frac{6\pi}{11})+\cos{(\frac{8\pi}{11})+\cos{(\frac{10\pi}{11})=-\frac{1}{2}}}}}\)

    \item If \(\alpha\) is a complex root of the equation \(z^5=1 \), show that \(\alpha+\alpha^2+\alpha^3+\alpha^4=-1\)
    
    \item The roots of the quadratic equation \(ax^2+bx+c=0\) are \(\sin{\theta}\text{ and } \cos{\theta}\).
    
    Show that: \(\frac{\sin{\theta}}{1-\cot{\theta}}+\frac{\cos{\theta}}{1-\tan{\theta}}=-\frac{b}{a}\)
\end{enumerate}
\end{document}