\documentclass[../main.tex]{subfiles}
\graphicspath{{\subfix{../images/}}}
\begin{document}


\section{Log problems}
You should be familiar with all of the log rules:

\(y=\log_b(x) \Longleftrightarrow x=b^y \)

\(\log_b(xy)=\log_b(x)+\log_b(y) \)

\(\log_b\bigl(\frac{x}{y}\bigr)=\log_b(x)-\log_b(y) \)

\(\log_b(x^n)=n\log_b(x) \)

\(\log_b(x)=\frac{\log_a(x)}{\log_a(b)} \)

When faced with tricky problems involving logs, we use the above rules to manipulate the equations into something we can solve more easily.
A common technique is to use the change of base formula to change a log term into a fraction with a different base. For example:

\(\log_8(x)+\log_16(x)=1 \)

Notice that both terms have bases which are powers of 2, therefore we will change the base to 2 for each term:

\(\frac{\log_2(x)}{\log_2(8)}=\frac{\log_2(x)}{3} \)

\(\frac{\log_2(x)}{\log_2(16)}=\frac{\log_2(x)}{4} \)

Giving us an equation of: \(\frac{\log_2(x)}{3}+\frac{\log_2(x)}{4}=1\)

We can then easily solve:

\(4\log_2(x)+3\log_2(x)=12\)

\(7\log_2(x)=12\)

\(\log_2(x)=\frac{12}{7}\)

\(x=2^{\frac{12}{7}}=3.28\)

Another technique is to take the log of both sides to help us rearrange the equation into something easier to solve. For example:

\(x^{\log_2(x)}=256x^2 \)

If we take $\log_2$ of both sides, we get:

\(\log_2(x^{\log_2(x)})=\log_2(256x^2)\)

We can now move the power on the LHS out to the front, and also split the RHS into two terms.

\(\log_2(x)\log_2(x)=\log_2(256)+\log_2(x^2)\)

Simplifying:

\((\log_2(x))^2=8+2\log_2(x)\)

This is a quadratic where the subject is $\log_2(x)$, so if we do a u-substitution where $u=\log_2(x)$ we get:

\(u^2-2u-8=0\)

Solving, we have $u=-2, 4$.

Now we just reverse our substitution to find the value(s) of x:

\(\log_2(x)=-2 \rightarrow x=\frac{1}{4}\)

\(\log_2(x)=4 \rightarrow x=16\)


\pagebreak

\subsection*{Questions}
(Answers - page {\pageref*{Log problems answers}})
\label{Log problems}

\begin{enumerate}[itemsep=1cm]
    \item Solve for x:
    
    \(x^{\log_3(x)}=81x^3\)
    
    \item Solve for x:
    
    \(\log_4(2^x + 48)=x-1\)

    \item 
    \(\log_x(y)+\log_y(x)=2 \)
    Find the value of \(\frac{x}{y}+\frac{y}{x}\)

    \item 
    If $\sqrt{\log_a(b)}+\sqrt{\log_b(a)}=2$, then find the value of $\log_{ab}(a)-\log_{\frac{1}{ab}}(b)$

    \item 
    If $2^{3x-5}=3^{x+3}$ and $x=\log(864^{\log_{10}(y)})$, then find the value of $y^{\log_{10}\frac{8}{3}}$

    \item 
    Solve for x:

    \(\log_7\bigl(\log_9(x^2+\sqrt{x+1}+8)\bigr)=0\)

    \item 
    If $\log_{16}(x)+\log_8(y)=11$ and $\log_8(x)+\log_{16}(y)=10$ then find the value of $\frac{y}{x^2}$

    \item 
    Solve for x:

    \(\log_{\log_2(x)}(4)=\log_2(\log_4(x)) \)

    \item 
    Solve for x and y:

    \(\log_4(x)+\log_9(y)=2\)

    \(\log_x(2)+\log_y(3)=1\)

    \item 
    If $\log_5(4), \log_5(2^x+\frac{1}{2})$ and $\log_5(2^x-\frac{1}{4})$ are in arithmetic progression, find the value of $x$ and also find the common difference.

    \item 
    Solve the system:

    \(\log_{10}(x^2+y^2)=1+\log_{10}(13)\)

    \(\log_{10}(x+y)-\log_{10}(x-y)=3\log_{10}(2)\)

    \item 
    Evaluate the expression:

    \(\frac{1}{1+\log_a(bc)}+\frac{1}{1+\log_b(ac)}+\frac{1}{1+\log_c(ab)} \)

\end{enumerate}





\end{document}