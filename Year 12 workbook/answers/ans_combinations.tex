\documentclass[../main.tex]{subfiles}
\graphicspath{{\subfix{../images/}}}
\begin{document}
\subsection*{Answers - Combinations and permutations (page \pageref{Combinations and permuations})}
\label{Combinations answers}
\begin{enumerate}
    \item
    \(^{10}C_2=\frac{10!}{2!\times 8!}=\frac{10\times 9}{2}=45\)
    \item 
        \begin{enumerate}
            \item 
            \(5!=5\times 4\times 3\times 2\times 1=120\)
            \item 
            Visualise this with the girls effectively being a sixth member of the group. There are \(6!\) ways of arranging them.

            Then, within the girls, there are \(3!\) ways of arranging them.

            This means there are \(6! \times 3!=720\times 6=4320\) possible photos.
        \end{enumerate}
    \item 
        \begin{enumerate}
            \item 
            \(6\times ^5C_2 \times ^3C_3 =6\times 10\times 1=60\)
            \item 
            \(^6C_2 \times ^4C_2 \times ^2C_2 = 15\times 6\times 1=90\)
        \end{enumerate}
    \item 
    \( ^{20}C_3\times ^{30}C_2 = 1140\times 435=495,900 \)
    \item 
    2 candidates: \(^8C_2 = 28\)

    1 candidate: \(^8C_1 = 8\)

    0 candidates  = 1

    Total = \(37\)
    \item 
    \(^{15}C_3 \times ^9C_1 \times ^7C_1 = 28,665\)
    \item 
    Consider the two situations: first, where all 6 people are from the same college. Second, where 4 are from the same college and 2 are from the other one.

    6 from same college: \(^8C_6=28\)

    4 from same college: \(^8C_4=70\)

    Total is \(98\)

    \item 
    Break into 3 situations: 

    \textbf{Situation 1}: all 3 sides are the same colour.

    There are 5 colours, so there are 5 ways this can occur.

    \textbf{Situation 2}: all 3 sides are different colours.

    We are fitting 5 colours into 3 spots, therefore \(^5C_3=10\)

    \textbf{Situation 3}: 2 sides have the same colour and one is different.

    

    \item 
    \begin{align*}
        \frac{p!}{q!(p-q)!}
        &=\frac{p!}{r!(p-r)!}\\
        \frac{1}{q!(p-q)!}
        &=\frac{1}{r!(p-r)!}\\
    \end{align*}
    There are 2 solutions to consider here. The first gives us the solution \(q=r \),which we are told is not a solution.

    \begin{align*}
        \frac{r!}{(p-q)!}
        &=\frac{q!}{(p-r)!}\\
    \end{align*}
    Here we can equate the numerators and the denominators, giving us \(r=q\).

    The other way is to cross-multiply different terms:

    \begin{align*}
    \frac{r!}{q!}=\frac{(p-q)!}{(p-r)!}
    \end{align*}
    When we equate the numerators and denominators we get:

    \(p-q=r\) and \(p-r=q\)

    Both of which can be rearranged to give the solution \(p=q+r\)
    \item 
    $
    \!
    \begin{aligned}[t]
    \frac{n!}{r!(n-r)!}
    &=\frac{(n+1)!}{(r-1)!((n+1)-(r-1))!}\\
    \frac{n!}{r!(n-r)!}
    &=\frac{(n+1)!}{(r-1)!(n-r+2)!}\\
    \frac{n!}{r!(n-r)!}
    &=\frac{(n+1)n!}{(r-1)!(n-r+2)(n-r+1)(n-r)!}\\
    \frac{1}{r!}
    &=\frac{n+1}{(r-1)!(n-r+2)(n-r+1)}\\
    \frac{(r-1)!}{r(r-1)!}
    &=\frac{n+1}{(n-r+2)(n-r+1)}\\
    \frac{1}{r}
    &=\frac{n+1}{(n-r+2)(n-r+1)}\\
    (n-r+2)(n-r+1)
    &=r(n+1)\\
    n^2 - rn + n -rn+r^2-r+2n-2r+2
    &=rn+r\\
    n^2-3rn+3n+r^2-4r+2
    &=0\\
    n^2+(3-3r)n+(r^2-4r+2)
    &=0\\
    n
    &=\frac{3r-3 \pm \sqrt{(3-3r)^2-4(r^2-4r+2)}}{2}\\
    n
    &=\frac{3r-3 \pm \sqrt{5r^2-2r+1}}{2}
    \end{aligned}
    $

    Now we try different values for r to see which gives an integer value for \(n\).
    
    $
    \begin{aligned}
    r=1; &n=1\\
    r=2; &n=\frac{3 \pm \sqrt{17}}{2}\\
    r=3; &n=\frac{6 \pm \sqrt{40}}{2}\\
    r=4; &n=\frac{9 \pm \sqrt{73}}{2}\\
    r=5; &n=\frac{12 \pm \sqrt{112}}{2}\\
    r=6; &n=\frac{15 \pm \sqrt{169}}{2}=\frac{15\pm 13}{2}=1,14
    \end{aligned}
    $

    \item
    \(k\frac{n!}{k!(n-k)!}=n\frac{(n-1)!}{(k-1)!((n-1)-(k-1))!} \)

    Note the following:

    \(n \times (n-1)! = n!\)

    \(k! = k \times (k-1)! \)

    Which means we can simplify the equation as follows:

    \(k\frac{n!}{k(k-1)!(n-k)!}=\frac{n!}{(k-1)!(n-k)!} \)

    \(k\frac{n!}{k(k-1)!(n-k)!}=\frac{n!}{(k-1)!(n-k)!} \)

    \(\frac{n!}{(k-1)!(n-k)!}=\frac{n!}{(k-1)!(n-k)!} \)

    \item 
    Firstly, note that from Pascal's Triangle, the sum of the numbers in the $n^{th}$ row is $2^n$.

    This means that $2^n = \bigl(^n_0\bigr)+\bigl(^n_1\bigr)+\bigl(^n_2\bigr)+\dots+\bigl(^n_n\bigr)$

    This means the $2^{n+1}$ term can be written as $\bigl(^{n+1}_0\bigr)+\bigl(^{n+1}_1\bigr)+\bigl(^{n+1}_2\bigr)+\dots+\bigl(^{n+1}_n\bigr)+\bigl(^{n+1}_{n+1}\bigr)$

    Since $\bigl(^{n+1}_0\bigr)=1$, we can write $2^{n+1}-1=\bigl(^{n+1}_1\bigr)+\bigl(^{n+1}_2\bigr)+\dots+\bigl(^{n+1}_n\bigr)+\bigl(^{n+1}_{n+1}\bigr)$

    The left-hand side of the equation refers to the $n^{th}$ row of Pascal's Triangle whereas the right-hand side refers to the $(n+1)^{th}$ row. We can now use the proof from the previous question to rewrite the RHS in terms of the $n^{th}$ row.

    We know that \(k\bigl(^n_k\bigr)=n\bigl(^{n-1}_{k-1}\bigr) \)

    This can be rearranged to \(\bigl(^n_k\bigr)=\frac{n}{k}\bigl(^{n-1}_{k-1}\bigr) \), and since we want to link rows $n$ and $n+1$ we rewrite it as \(\bigl(^{n+1}_{k}\bigr)=\frac{n+1}{k}\bigl(^{n}_{k-1}\bigr) \)

    Now, each term in the expansion of $2^{n+1}-1$ can be rewritten in terms of row $n$:

    \(\frac{n+1}{1}\bigl(^n_0\bigr)+\frac{n+1}{2}\bigl(^n_1\bigr)+\frac{n+1}{3}\bigl(^n_2\bigr)+\dots+\frac{n+1}{n}\bigl(^n_{n-1}\bigr)+\frac{n+1}{n+1}\bigl(^n_n\bigr)\)

    Returning to the original RHS, \(\frac{2^{n+1}-1}{n+1}\), we can divide out the $n+1$, giving us \(\bigl(^n_0\bigr)+\frac{1}{2}\bigl(^n_1\bigr)+\frac{1}{3}\bigl(^n_2\bigr)+\dots+\frac{1}{n}\bigl(^n_{n-1}\bigr)+\frac{1}{n+1}\bigl(^n_n\bigr)=LHS\)
\end{enumerate}

\end{document}