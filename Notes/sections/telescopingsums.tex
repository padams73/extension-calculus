\documentclass[../main.tex]{subfiles}
\graphicspath{{\subfix{../images/}}}
\begin{document}
\section{Telescoping Sums}
Telescoping series involve long sums where patterns can enable us to do mass cancellations, making the problem easily solvable.

For example, the sum $S=1-\frac{1}{2}+\frac{1}{2}-\frac{1}{3}+\frac{1}{3}+...+\frac{1}{99}-\frac{1}{100}$

This should be relatively obvious as you can quickly see that all terms other than the first and last will cancel out:

$S=1-\frac{1}{100}=\frac{99}{100}$

Of course, these are never this straight-forward! The trick is usually spotting the pattern.

\subsection*{Factoring}
This can be used in examples like below, where each denominator can be written as the product of two factors that always have the same difference.

$S=\frac{1}{4}+\frac{1}{28}+\frac{1}{70}+\dots +\frac{1}{9700}$

Notice that we can write this sum as $S=\frac{1}{1\times 4}+\frac{1}{4\times 7}+\frac{1}{7\times 10}+\dots +\frac{1}{97\times 100}$

Since each factor pair differs by 3, we can write the sum this way:

$S=\frac{1}{3}\Bigl(\frac{4-1}{1\times 4}+\frac{7-4}{4\times 7}+\frac{10-7}{7\times 10}+\dots +\frac{100-97}{97\times 100}\Bigr)$

$S=\frac{1}{3}\Bigl(\frac{4}{4}-\frac{1}{4}+\frac{7}{28}-\frac{4}{28}+\frac{10}{70}-\frac{7}{70}+\dots +\frac{100}{9700}-\frac{97}{9700}\Bigr)$

$S=\frac{1}{3}\Bigl(\frac{1}{1}-\frac{1}{4}+\frac{1}{4}-\frac{1}{7}+\frac{1}{7}-\frac{1}{10}+\dots +\frac{1}{97}-\frac{1}{100}\Bigr)$

$S=\frac{1}{3}\Bigl(1-\frac{1}{100}\Bigr)$

$S=\frac{1}{3}\times \frac{99}{100}=\frac{33}{100}$

\subsection*{Rationalising}
By rationalising fractions with surds in the denominator, then simplifying, we may find that terms cancel out. This can occur when the second surd in the denominator of a term is same as the first surd in the denominator of the following term.

$S=\frac{1}{\sqrt{2}+\sqrt{5}}+\frac{1}{\sqrt{5}+\sqrt{8}}+\frac{1}{\sqrt{8}+\sqrt{11}}+\dots +\frac{1}{\sqrt{98}+\sqrt{101}}$

$S=\frac{1}{\sqrt{2}+\sqrt{5}}\times \frac{\sqrt{2}-\sqrt{5}}{\sqrt{2}-\sqrt{5}}+\frac{1}{\sqrt{5}+\sqrt{8}}\times \frac{\sqrt{5}-\sqrt{8}}{\sqrt{5}-\sqrt{8}}+\frac{1}{\sqrt{8}+\sqrt{11}}\times \frac{\sqrt{8}-\sqrt{11}}{\sqrt{8}-\sqrt{11}}+\dots +\frac{1}{\sqrt{98}+\sqrt{101}}\times \frac{\sqrt{98}-\sqrt{101}}{\sqrt{98}-\sqrt{101}}$

$S=\frac{\sqrt{2}-\sqrt{5}}{-3}+\frac{\sqrt{5}-\sqrt{8}}{-3}+\frac{\sqrt{8}-\sqrt{11}}{-3}+\dots +\frac{\sqrt{98}-\sqrt{101}}{-3}$

$S=-\frac{\sqrt{2}}{3}+\frac{\sqrt{5}}{3}-\frac{\sqrt{5}}{3}+\frac{\sqrt{8}}{3}-\frac{\sqrt{8}}{3}+\frac{\sqrt{11}}{3}-\dots -\frac{\sqrt{98}}{3}+\frac{\sqrt{101}}{3}$

$S=\frac{\sqrt{101}-\sqrt{2}}{3}$

\subsection*{Partial fractions}
Partial fractions can often be useful in helping us to find the patterns. By splitting a denominator with a product into two separate fractions, we sometimes find the fractions will cancel out.

For example:

\(\sum\limits_{x=1}^\infty \frac{1}{x(x+3)}\)

Using partial fraction decomposition:

\(\sum\limits_{x=1}^\infty \frac{1}{x(x+3)}=\Bigl(\frac{1}{3x}-\frac{1}{3x+9}\Bigr)\)

Setting up the series by substituting values of $x$ from 1 up to infinity:

$S=\frac{1}{3}-\frac{1}{12}+\frac{1}{6}-\frac{1}{15}+\frac{1}{9}-\frac{1}{18}+\frac{1}{12}-\frac{1}{21}+\frac{1}{15}-\frac{1}{24}+\dots +\frac{1}{\infty}-\frac{1}{\infty}$

You can see that all terms except for $\frac{1}{3}$, $\frac{1}{6}$ and $\frac{1}{9}$ will cancel out. The terms eventually become infinitely small as the denominator becomes infinitely large, so they effectively become zero and do not affect the sum.

Therefore, the sum is $S=\frac{1}{3}+\frac{1}{6}+\frac{1}{9}=\frac{11}{18}$

\pagebreak
\hypertarget{telescopingsumslink}{\subsection*{Questions}}
\hyperlink{telescopingsumsanswers}{(Answers - page {\pageref*{Telescoping sums answers}})}

\label{telescoping sums}
\begin{enumerate}[itemsep=1cm]
    \item 
    Evaluate $\sum\limits_{n=1}^\infty \Bigl[\frac{1}{n+1}-\frac{1}{n+2}\Bigr]$

    \item 
    Evaluate: $\frac{1}{1+\sqrt{2}}+\frac{1}{\sqrt{2}+\sqrt{3}}+\frac{1}{\sqrt{3}+\sqrt{4}}+\dots +\frac{1}{\sqrt{99}+\sqrt{100}}$

    \item 
    Find the value of the sum:

    $\frac{1}{3+\sqrt{11}}+\frac{1}{\sqrt{11}+\sqrt{13}}+\frac{1}{\sqrt{13}+\sqrt{15}}+\dots +\frac{1}{\sqrt{10001}+\sqrt{10003}}$

    \item 
    Evaluate $\sum\limits_{n=1}^\infty \frac{1}{n(n+1)}$

    \item 
    Evaluate $\sum\limits_{n=1}^\infty \frac{1}{n^2 +4n+3}$

    \item 
    Evaluate $\sum\limits_{n=1}^{2015} \frac{1}{n^2 +3n+2}$

    \item 
    Evaluate: $\frac{1}{2^2-1}+\frac{1}{4^2-1}+\frac{1}{6^2-1}+\frac{1}{8^2-1}+\dots +\frac{1}{1000^2-1}$

    \item 
    Evaluate: $\frac{3}{4}+\frac{3}{28}+\frac{3}{70}+\frac{3}{130}+\dots +\frac{3}{9700}$

\end{enumerate}
\end{document}