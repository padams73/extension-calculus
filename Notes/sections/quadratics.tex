\documentclass[../main.tex]{subfiles}
\graphicspath{{\subfix{../images/}}}
\begin{document}
\section{Turning equations into quadratics}
When there are three terms in an equation, we can often turn them into a quadratic, where the subject is not \(x\) but another expression that we substitute in.\\
For example, \(e^{4x}-5e^{2x}+6=0\) can be solved by making it a quadratic in terms of \(e^{2x}\).\\

\(u=e^{2x}\)\\
\(u^2-5u+6=0\)\\
\(u=2, 3\)\\
Then we just back-substitute and solve:\\
\(
e^{2x}=2     \\      
2x=\ln{2}      \\ 
x=\frac{\ln{2}}{2}\\     
e^{2x}=3\\
2x=\ln{3}\\
x=\frac{\ln{3}}{2}
\)\\

If all three terms contain a variable, we can also divide the equation through by something to turn one of those into a constant, enabling us to then solve it as a quadratic.\\
For example, \(3(2^{3x})-11(2^{2x})-2^{x+2}=0\)\\
If we divide each term by a common factor of \(2^x\), the equation changes to:\\

\(\frac{3(2^{3x})}{2^x}-\frac{11(2^{2x})}{2^x}-\frac{2^{x+2}}{2^x}=0\)\\

\(3(2^{2x})-11(2^x)-2^2=0\)\\

We can now make the substitution \(u=2^x\) to solve the equation:\\

\(3u^2-11u-4=0\)\\
\(u=-\frac{1}{3}, 4\)\\

Since \(2^x\) can clearly never be negative, we can disregard the first solution.\\
\(2^x=4\)\\
\(x=2\)\\

\pagebreak

\subsection*{Questions}
\label{quadratics}
\begin{enumerate}
    \item Solve \(2^x+4^x=24\)
    \item Solve \(4^x+6^x=9^x\)
    \item Solve \(8(9^x)+3(6^x)-81(4^x)=0\)
    \item Solve \(25^x+2(15^x)-24(9^x)=0\)
\end{enumerate}
\end{document}