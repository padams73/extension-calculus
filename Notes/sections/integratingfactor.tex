\documentclass[../main.tex]{subfiles}
\graphicspath{{\subfix{../images/}}}
\begin{document}


\section{Integrating factor method}
Not every differential equation can be solved by separation of variables.

When a differential equation is in the general form of $\frac{dy}{dx}+p(x)y=q(x)$, we can use a method called the integrating factor.

The integrating factor is defined as $\mu=e^{\int p(x)\,dx}$

We multiply both sides of the equation by this, to get:

\[\mu \frac{dy}{dx}+\mu.p(x)y=\mu.q(x)\]

\textit{Why do we do this?}

This is helpful because if we consider the product $\mu y$, and look at the derivative:

\[\frac{d}{dx}(\mu y)=\frac{d\mu}{dx}y+\mu\frac{dy}{dx}\]

We can find $\frac{d\mu}{dx}$ from our earlier definition:

\[\frac{d\mu}{dx}=\frac{d}{dx}\Bigl(e^{\int p(x)\,dx}\Bigr)\]

By the Chain Rule we get:

\[\frac{d\mu}{dx}=p(x).e^{\int p(x)\,dx}=\mu.p(x)\]

All of this means that the left-hand side is now the same as the derivative of the integrating factor multiplied by $y$.

i.e. $\frac{d}{dx}\Bigl(e^{\int p(x)}\,dx \times y\Bigr)=e^{\int p(x)\,dx}\frac{dy}{dx}+e^{\int p(x)\,dx}p(x)y=\mu \frac{dy}{dx}+\mu p(x)y$

This means we can rewrite the equation as:

\[\frac{d}{dx}\Bigl(\mu y\Bigr)=\mu.q(x)\]

Which we can solve by direct integration.


\pagebreak
\subsection*{Example}
Solve the differential equation $x\frac{dy}{dx}+3xy=xe^x$

Start by dividing through by $x$ to put the equation into standard form.

\[\frac{dy}{dx}+3y=e^x\]

From this we identify that $p(x)=3$ and $q(x)=e^x$

Next we define the integrating factor $\mu=e^{\int 3\,dx}=e^{3x}$

Multiplying through by the integrating factor:

\[e^{3x}\frac{dy}{dx}+3e^{3x}y=e^{3x}.e^x\]

\[e^{3x}\frac{dy}{dx}+3e^{3x}y=e^{4x}\]

Consider that $\frac{d}{dx}e^{3x}y=e^{3x}\frac{dy}{dx}+3e^{3x}y$ which is the same as the left side of the equation. We can rewrite the equation as:

\[\frac{d}{dx}\Bigl(e^{3x}y\Bigr)=e^{4x}\]

We can now integrate both sides and rearrange to solve:

\[\int \frac{d}{dx}\Bigl(e^{3x}y\Bigr)=\int e^{4x}\,dx\]

\[e^{3x}y=\frac{e^{4x}}{4}+c\]

\[y=\frac{e^x}{4}+ce^{-3x}\]

\pagebreak

\subsection*{Questions}
(Answers - {\pageref{Integrating factor answers}})
\label{Integrating factor method}

Use the integrating factor method to solve the differential equations. You can find the value of the constant by using the given coordinates.
\begin{enumerate}[itemsep=0.7cm]
    \item 
    $\frac{dy}{dx}+2y=4; y(0)=4$

    \item 
    $\frac{dy}{dx}+2y=e^{4x}; y(0)=4$

    \item 
    $\frac{dy}{dx}+y=e^{-x}; y(0)=1$

    \item 
    $\frac{dy}{dx}+2xy=x; y(1)=1$

    \item 
    $\frac{dy}{dx}+3x^2 y=e^{x-x^3}; y(0)=2$

    \item 
    $4\frac{dy}{dx}+y=3x; y(2)=6$

    \item 
    $x\frac{dy}{dx}+y=1; x>0, y(1)=1$

    \item 
    $x\frac{dy}{dx}+5y=\frac{3}{x^5 \ln{(x)}}; x\geq e; y(e)=1$

    \item 
    $2\frac{dy}{dx}+4xy=(x+1)e^{2x}; y(e)=e$

    \item 
    $3\frac{dy}{dx}-3\sin{(2x)}y=e^{-\cos^2{(x)}}; y\Bigl(\frac{3\pi}{2}\Bigr)=\pi$



\end{enumerate}





\end{document}