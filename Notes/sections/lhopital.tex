\documentclass[../main.tex]{subfiles}
\graphicspath{{\subfix{../images/}}}
\begin{document}

\section{Evaluating limits}

\textbf{Defined at the value}\\
A limit tells us how a function behaves as it approaches a value. When the function is defined at the value such as $\displaystyle \lim_{x\rightarrow 2} x^2=2^2=4$

\textbf{Not defined at the value}\\
If the function is not defined at the value, such as $\displaystyle \lim_{x\rightarrow 2} \frac{x^2-5x+6}{x-2}$, we can try to simplify the function.

In this case, we can rewrite the limit as: $\displaystyle \lim_{x\rightarrow 2} \frac{(x-2)(x-3)}{x-2}=\lim_{x\rightarrow 2} x-3=-1$

\textbf{Limits as $x\rightarrow \infty$}\\
When we are finding the limit of a rational fraction with $x\rightarrow \infty$, we can divide every term by the highest power, making many of the terms go to zero.

For example, $\displaystyle \lim_{x\rightarrow \infty} \frac{2x^4-x^3}{3x^4+x^2-x}$

Here we can divide each term by $x^4$, giving us: $\displaystyle \lim_{x\rightarrow \infty} \frac{2-\frac{1}{x}}{3+\frac{1}{x^2}-\frac{1}{x^3}}=\frac{2-0}{3+0-0}=\frac{2}{3}$


\subsection*{L'H\^{o}pital's Rule for indeterminate cases}
\setstretch{1.5}
L'H\^{o}pital's Rule is a technique used for dealing with limits that involve \textit{indeterminate} forms.

\textit{Indeterminate} in this case means that the result is either $\frac{0}{0}$ or $\frac{\infty}{\infty}$.

Consider limits where both the numerator and denominator both approach zero as $x \rightarrow a$.

For example, $\displaystyle \lim_{x\rightarrow -2} \frac{x^2-4}{x+2}$, or $\displaystyle \lim_{x\rightarrow 0}\frac{\sin{x}}{x}$

\setstretch{1}
In the situation where:

\begin{itemize}
    \item $f(x)$ and $g(x)$ are continuous
    \item $f'(x)$ and $g'(x)$ are continuous
    \item $\displaystyle \lim_{x \rightarrow a}f(x)=0$
    \item $\displaystyle \lim_{x \rightarrow a}g(x)=0$
\end{itemize} 

    \[\lim_{x \rightarrow a}\frac{f(x)}{g(x)}=\lim_{x \rightarrow 0}\frac{f'(x)}{g'(x)}\]

Provided the last limit exists or is $\pm \infty$

Similarly, if $\displaystyle \lim_{x \rightarrow a}f(x)=\infty$ and $\displaystyle \lim_{x \rightarrow a}g(x)=\infty$, then:

\[\lim_{x \rightarrow a}\frac{f(x)}{g(x)}=\lim_{x \rightarrow \infty}\frac{f'(x)}{g'(x)}\]

Put simply, if you get an indeterminate limit, you can differentiate the numerator and denominator and then take the limit again.

Note: if after applying L'H\^{o}pital you get another indeterminate limit, you can apply it again.


\subsection*{Examples}
\begin{enumerate}[itemsep=1cm]
    \item
    $\displaystyle \lim_{x\rightarrow 0}\frac{2x^3+x}{x^2-x}=\frac{0}{0}$

    Therefore, by applying L'H\^{o}pital, we get $\displaystyle \lim_{x\rightarrow 0}\frac{6x^2+1}{2x-1}=\frac{1}{-1}=-1$

    \item
    $\displaystyle \lim_{x\rightarrow \infty}\frac{x^2}{e^x}=\frac{\infty}{\infty}$

    By applying L'H\^{o}pital, we get:

    $\displaystyle \lim_{x\rightarrow \infty} \frac{2x}{e^x}=\frac{\infty}{\infty}$

    We apply L'H\^{o}pital a second time:

    $\displaystyle \lim_{x\rightarrow \infty} \frac{2}{e^x}=\frac{2}{\infty}=0$
\end{enumerate}
\pagebreak

\subsection*{Questions}
(Answers - page \pageref*{L'hopital answers})
\label{L'hopital}

Find the limits:

\begin{enumerate}[itemsep=0.7cm]
    \item 
    $\displaystyle \lim_{x\rightarrow 1}\frac{x^2+2x-3}{x^2-6x+5}$

    \item 
    $\displaystyle \lim_{x\rightarrow -2}\frac{x^2-4}{x+2}$

    \item 
    $\displaystyle \lim_{x\rightarrow \infty}\frac{2x^3-3x^2}{x^4+3x^2}$

    \item 
    $\displaystyle \lim_{x\rightarrow 0}\frac{\sin{x}}{x}$

    \item 
    $\displaystyle \lim_{x\rightarrow 0}\frac{\tan{x}}{\sin{x}}$

    \item 
    $\displaystyle \lim_{x\rightarrow 0}\frac{x-\sin{x}}{1-\cos{x}}$

    \item 
    $\displaystyle \lim_{x\rightarrow 0}\frac{3x^2+x^3}{x^2+x^4}$

    \item 
    $\displaystyle \lim_{x\rightarrow \infty}\frac{3x^2+x^3}{x^2+x^4}$

    \item 
    $\displaystyle \lim_{x\rightarrow \infty}\frac{e^x}{x^2}$

    \item 
    $\displaystyle \lim_{x\rightarrow \infty}2x\sin{\frac{\pi}{x}}$

    \item 
    $\displaystyle \lim_{x\rightarrow \infty}x e^{-x}$

\end{enumerate}




\end{document}