\documentclass[../main.tex]{subfiles}
\graphicspath{{\subfix{../images/}}}
\begin{document}


\section{Trigonometric identities}
We can use combinations of the standard trigonometric identities given in the formula sheet to prove more complex identities.

The most common identities you will use are below.

Compound angle rules:

\begin{equation*}
    \begin{aligned}
\sin{(A\pm B)}&=\sin{A}\cos{B}\pm \cos{A}\sin{B}\\
\cos{(A\pm B)}&=\cos{A}\cos{B}\mp \sin{A}\sin{B}\\
\tan{(A\pm B)}&=\frac{\tan{A}\pm \tan{B}}{1 \mp \tan{A}\tan{B}}
\end{aligned}
\end{equation*}

Double angle rules:
\begin{equation*}
    \begin{aligned}
\sin{(2A)}&=2\sin{A}\cos{A}\\
\tan{(2A)}&=\frac{2\tan{A}}{1-\tan^2{A}}\\
    \cos{(2A)}&=\cos^2{A}-\sin^2{A}\\
    &=2\cos^2{A}-1\\
    &=1-2\sin^2{A}
\end{aligned}
\end{equation*}

Identities:

\[\cos^2{\theta}+\sin^2{\theta}=1\]
\[\tan^2{\theta}+1=\sec^2{\theta}\]
\[\cot^2{\theta}+1=\csc^2{\theta}\]


The best way to do this is to start on one side and transform it so that it is shown to be equivalent to the other side.
 
In general, start with the more complex side, as it is easier to simplify something complex than it is to complicate something simple.

E.g. prove that $\sin{\theta}(1+\tan{\theta})+\cos{\theta}(1+\cot{\theta}) \equiv \sec{\theta}+\csc{\theta}$

Start with the LHS as it is more complicated.

$\sin{\theta}(1+\tan{\theta})+\cos{\theta}(1+\cot{\theta})$

Looking at the RHS, we can see that we need to get sec and cosec.

We can change the first term by multiplying the $\sin{\theta}$ by $\sin{\theta}$ while dividing each of the terms in the bracket by $\sin{\theta}$.

$\sin^2{\theta}\Bigl(\frac{1}{\sin{\theta}}+\frac{\tan{\theta}}{\sin{\theta}}\Bigr)=\sin^2{\theta}(\csc{\theta}+\sec{\theta})$

We can repeat this for the second term, using $\cos{\theta}$ instead.

$\cos^2{\theta}\Bigl(\frac{1}{\cos{\theta}}+\frac{\cot{\theta}}{\cos{\theta}}\Bigr)=\cos^2{\theta}(\sec{\theta}+\csc{\theta})$

So the LHS now looks like this:

\(\sin^2{\theta}(\csc{\theta}+\sec{\theta})+\cos^2{\theta}(\sec{\theta}+\csc{\theta})\)

Factorising gives us:

$(\sin^2{\theta}+\cos^2{\theta})(\sec{\theta}+\csc{\theta})$

Using the Pythagorean identity of $\sin^2{\theta}+\cos^2{\theta}=1$, we get $\sec{\theta}+\csc{\theta}=RHS$, as required.

\vspace{5mm}
Another example:

Show that $\tan{A}+\cot{A}=\frac{1}{\sin{A}\cos{A}}$

Using $\tan{A}=\frac{\sin{A}}{\cos{A}}$ and $\cot{A}=\frac{1}{\tan{A}}$

LHS = $\tan{A}+\cot{A}=\frac{\sin{A}}{\cos{A}}+\frac{\cos{A}}{\sin{A}}$

$=\frac{\sin^2{A}+\cos^2{A}}{\sin{A}\cos{A}}$

Since we know that $\sin^2{A}+\cos^2{A}=1$

$=\frac{1}{\sin{A}\cos{A}}=RHS$, as required.

\vspace{5mm}
An example of working with both sides:

Show that: $\frac{\sin{A}-\cos{B}}{\sin{B}-\cos{A}}=\frac{\cos{A}+\sin{B}}{\cos{B}+\sin{A}}$

Multiplying the equation by $\cos{B}+\sin{A}$

$\frac{\sin^2{A}-\cos^2{B}}{\sin{B}-\cos{A}}=\cos{A}+\sin{B}$

Multiplying the equation by $\sin{B}-\cos{A}$

$\sin^2{A}-\cos^2{B}=\sin^2{B}-\cos^{A}$

Rearranging:

$\sin^2{A}+\cos^{A}=\sin^2{B}+\cos^2{B}$

$1=1$

Since this is a true statement, we have shown the original equation is always true.

\pagebreak

\subsection*{Questions} 
\label{Trig identities}
(Answers - page {\pageref{Trig identities answers}})

Easier questions:

For each of the following, show that:
\begin{enumerate}
    \item $\frac{\sin{A}+\cos{A}}{\sin{A}-\cos{A}}=\frac{1+2\cos{A}\sin{A}}{1-2\cos^2{A}}$
    
    \item $\frac{\sin{2A}}{1+\cos{2A}}=\tan{A}$
    
    \item $\sin{2A}=\frac{2\tan{A}}{1+\tan^2{A}}$
    
    \item $\frac{\sin{2A}}{\sin{A}}-\frac{\cos{2A}}{\cos{A}}=\sec{A}$
    
    \item $(\sec{A}-\tan{A})^2=\frac{1-\sin{A}}{1+\sin{A}}$
    
    \item $\tan{A}=\sqrt{\frac{1-\cos{2A}}{1+\cos{2A}}}$

    \item $\frac{\csc^2{A}-1}{\cos^2{A}}+\frac{1}{1-\sin^2{A}}=\sec^2{A}\csc^2{A}$
    
    \item $\frac{\cos{A}}{1+\sin{A}}=\frac{1-\sin{A}}{\cos{A}}$
    
    \item $2\csc{4A}+2\cot{4A}=\cot{A}-\tan{A}$
    
    \item $\frac{\sin{3A}}{\sin{2A}-\sin{A}}=2\cos{A}+1$
    
    \item $\frac{1+\cos{A}}{1-\cos{A}}=(\csc{A}+\cot{A})^2$
    
    \item $\cos{2A}=\frac{1-\tan^2{A}}{1+\tan^2{A}}$
    
    \item $\cos{3A}=4\cos^3{A}-3\cos{A}$
    
    \item $\cos{4A}=1-8\sin^2{A}\cos^2{A}$
    
    \item $\tan{3A}=\frac{3\tan{A}-\tan^3{A}}{1-3\tan^2{A}}$
    
    \item $\tan{4A}=\frac{4\tan{A}-4\tan^3{A}}{1-6\tan^2{A}+\tan^4{A}}$
    
    \item $4\sin^3{A}\cos{3A}+4\cos^3{A}\sin{3A}=3\sin{4A}$


\end{enumerate}
Harder problems (including old scholarship questions):
\begin{enumerate}
    \setcounter{enumi}{18}
    \item $\frac{\csc{A}-\cot{A}}{\csc{A}+\cot{A}}+\frac{\csc{A}+\cot{A}}{\csc{A}-\cot{A}}\equiv 2+4\cot^2{A}$
    
    \item $\frac{1-\sin{A}}{1-\sec{A}}-\frac{1+\sin{A}}{1+\sec{A}}\equiv 2\cot{A}(\cos{A}-\csc{A})$
    
    \item $\frac{1+\cos{A}}{1-\cos{A}}\equiv (\csc{A}+\cot{A})^2$
    
    \item $\frac{\sin{(\pi-B)}-\sin{A}}{\cos{A}+\cos{(\pi - B)}}\equiv \frac{\cos{A}+\cos{B}}{\sin{B}+\sin{(\pi - A)}}$
    
    \item $\frac{\csc{A}-\sec{A}}{\csc{A}+\sec{A}}(\cot{A}-\tan{A})\equiv \sec{A}\csc{A}-2$
    
    \item $(\sec{A}-2\sin{A})(\csc{A}+2\cos{A})\sin{A}\cos{A}\equiv (\cos^2{A}-\sin^2{A})^2$
    
    \item 2018 Scholarship exam:
    
    $\frac{\cos{\theta}}{1+\sin{\theta}}-\frac{\sin{\theta}}{1+\cos{\theta}}=\frac{2(\cos{\theta}-\sin{\theta})}{1+\sin{\theta}+\cos{\theta}}$

    \item 2017 Scholarship exam:
    
    $\cos{(5\theta)}=16\cos^5{\theta}-20\cos^3{\theta}+5\cos{\theta}$
    
\end{enumerate}



\end{document}