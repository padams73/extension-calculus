\documentclass[../main.tex]{subfiles}
\graphicspath{{\subfix{../images/}}}
\begin{document}


\section{Combinations and permutations}
Both of these refer to various ways in which objects from a set may be selected, generally without replacement, to form subsets.\\
A Permutation refers to selecting a subset where the order of selection matters, while a Combination is when the order does not matter.\\
In other words, Combinations are counting the how many selections we can make from \(n\) objects, while Permutations count the number of arrangements of \(n\) objects.\\
The formulas for each are below, where \(n\) is the number of objects and \(r\) is the size of the subset:\\

Permutations: \(^nP_r=\frac{n!}{(n-r)!}\)\\

Combinations: \(^nC_r=(_r^n)=\frac{n!}{r!(n-r)!}\)\\

\noindent E.g. If there are 20 people in a room and they all shake hands with each other, how many handshakes are there?\\
In this case, we are asking how many different subsets of size 2 can we select from a group of 20?\\
Since the order doesn't matter, as person A shaking hands with person B is the same as person B shaking hands with person A, we use the \textit{Combination} equation.\\

\((_2^{20})=\frac{20!}{2!(20-2)!}=\frac{20!}{2\times 18!}=\frac{20\times19}{2}=190\)\\

Notice that we can cancel out parts of the factorials since they have common factors, so that\\

\(\frac{20!}{18!}=\frac{20\times 19\times ... \times 2 \times 1}{18 \times 17 \times ... \times 2 \times 1} = 20 \times 19\)\\

E.g. If I want to select a Cantamaths team of 4 students from a class of 16, how many different teams are possible?\\

Again, since the order in not important (team ABCD is the same as team BADC), we use a combination.\\

\((_4^16)=\frac{16!}{4!(16-4)!}=\frac{16!}{4!\times 12!}=\frac{16\times 15\times 14\times 13}{4\times 3\times 2\times 1}=1820\)
\pagebreak

\subsection*{Questions}
\label{Combinations and permuations}
\begin{enumerate}
    \item 
    If there are 10 different people in a room and they all shake each other’s hands, how many handshakes are there?
    \item 
        \begin{enumerate}
            \item 5 boys stand in a line, posing for a photo. How many possible orders are there?
            \item 3 girls then join the group. How many possible photos are there if the girls must stand next to each other?
        \end{enumerate}
    \item 
    We have 6 books to distribute to three students A, B and C.\\
    How many different ways are there of distributing these books if:
        \begin{enumerate}
            \item A is given 1 book, B is given 2 books, and C is given 3 books?
            \item Each student is given 2 books?
        \end{enumerate}
    \item 
    A company has 20 male employees and 30 female employees. A grievance committee is to be established. If the committee will have 3 male employees and 2 female employees, how many ways can the committee be chosen?
    \item 
    Eight candidates are competing to get a job at a prestigious company. The company has the freedom to choose as many as two candidates. In how many ways can the company choose two or fewer candidates.
    \item 
    A committee of 5 members must be chosen from a track club. The club has 15 sprinters, 9 jumpers, and 7 long-distance runners. The committee must have exactly 1 jumper and 1 long-distance runner. How many ways can the committee be chosen?
    \item 
    There are 10 people forming a commission. Two of them are students from different colleges. The commission is composed of 6 members and if one of the students is in it the other must be as well. How many commissions like these can there be?
    \item 
    Using 3 sticks of 5 different colours, how many unique equilateral triangles can be made. Assume you have at least 3 sticks of each colour. Note: if a triangle can be rotated and/or flipped to create another, they are not different.
    \item 
    Given \(^pC_q=^pC_r, q\neq r\), express \(p\) in terms of \(q\) and \(r\).
    \item 
    There are many integer solutions to the equation \((^n_r)=(^{n+1}_{r-1}),\text{ including } n=r=1 \)//
    Find an expression for \(n\) in terms of \(r\), and hence find another of the integer solutions.
\end{enumerate}
\end{document}