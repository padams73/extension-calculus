\documentclass[../main.tex]{subfiles}
\graphicspath{{\subfix{../images/}}}
\begin{document}
\hypertarget{taylorseriesanswers}{\subsection*{Answers - Taylor series (\hyperlink{taylorserieslink} {page \pageref{taylor series}})}}

\label{Taylor series answers}
\begin{enumerate}[itemsep=0.7cm]
    \item 
    Derive the first two terms of the Taylor series to approximate the sine function about zero.

    $f(x) = \sin{(x)} \rightarrow f(0)=0$

    $f'(x) = \cos{(x)} \rightarrow f'(0)=1$

    $f''(x) = -\sin{(x)} \rightarrow f''(0)=0$
    
    $f^{(3)}(x) = -\cos{(x)} \rightarrow f^{(3)}(x)=-1$

    $p(0) = f(0) \rightarrow c_0 = 0$

    $p'(0) = f'(0) \rightarrow c_1 = 1 \therefore c_1 = 1$

    $p''(0) = f''(0) \rightarrow 2c_2 = 0 \therefore c_2 = 0$

    $p^{(3)}(0) = f^{(3)}(0) \rightarrow 6c_3 = -1 \therefore c_3 = -\frac{1}{6}$

    This gives the first two terms as $p(x) = x - \frac{x^3}{6}$

    \item 
    Derive the next two terms of this series, then generalise this as a sum.

    The derivatives of $f(x)$ rotate around, so we know that:
    \begin{itemize}
        \item     $p^{(4)}(0)=0$, so $c_4 = 0$
        \item     $p^{(5)}(0)=1$, so $c_5 = \frac{1}{5!}$
        \item     $p^{(6)}(0)=0$, so $c_6 = 0$
        \item     $p^{(7)}(0)=-1$, so $c_7 = -\frac{1}{7!}$  
    \end{itemize}

    This gives our polynomial as $\sin{(x)} = p(x) = x - \frac{x^3}{3!} + \frac{x^5}{5!} - \frac{x^7}{7!} + \dots$

    Generalising as a sum we get:
    $\sin{(x)}= \sum_{n=0}^{\infty} \frac{(-1)^n x^{(2n+1)}}{(2n+1)!}$

    \item
    Derive the Taylor series for the function $f(x)=e^x$ about zero, finding the first six terms and generalising.

    Since $f(x)=e^x$ differentiates to itself, we know that $f'(x)=e^x, f''(x)=e^x$, and so on. This also means that $f(0)=1, f'(0)=1$, and so on.

    To find the first term:
    \[p(0)=f(0)\]
    \[c_0=1\]
    To find the second term:
    \[p'(0)=c_1 + 2c_2(0) + 3c_3(0)^2 + 4c_4(0)^3 + 5c_5(0)^4 + 6c_6(0)^5 + \cdots = 1\]
    \[c_1 = 1\]
    To find the third term:
    \[p''(0)= 2c_2 + 2\times 3c_3(0) + 3\times 4c_4(0)^2 + 4\times 5c_5(0)^3 + 5\times 6c_6(0)^4 + \cdots = 1\]
    \[c_2 = \frac{1}{2!} = \frac{1}{2}\]
    To find the fourth term:
    \[p^{(3)}(0) = 2\times 3c_3 + 2\times 3\times 4c_4(0) + 3\times 4\times 5c_5(0)^2 + 4\times 5\times 6c_6(0)^3 + \cdots =1\]
    \[c_3 = \frac{1}{3!} = \frac{1}{6}\]
    Fifth term:
    \[p^{(4)}(0) = 24c_4 + 2\times 3\times 4\times 5c_5(0) + 3\times 4\times 5\times 6c_6(0)^2 + \cdots =1\]
    \[c_4 = \frac{1}{4!} = \frac{1}{24}\]
    Sixth term:
    \[p^{(5)}(0) = 120c_5 + 2\times 3\times 4\times 5\times 6c_6(0) + \cdots = 1\]
    \[c_5 = \frac{1}{5!} = \frac{1}{120}\]

    Therefore, the Taylor series for $e^x$ about $x=0$ is:
    \[p(x) = 1 + x + \frac{x^2}{2} + \frac{x^3}{6} + \frac{x^4}{24} + \frac{x^5}{120} + \cdots\]

    Generalising the sum:
    \[e^x = p(x) = \sum_{n=0}^{\infty} \frac{x^n}{n!}\]

    \item
    Substitute $x=i\theta$ into the Taylor series for $e^x$ to show that $z=\cos{(\theta)}+i \sin{(\theta)}$ can also be written as $z=e^{i \theta}$

    \[p(i\theta) = 1 + i\theta + \frac{(i\theta)^2}{2} + \frac{(i\theta)^3}{6} + \frac{(i\theta)^4}{24} + \frac{(i\theta)^5}{120} + \cdots\]
    \[p(i\theta) = 1 + i\theta - \frac{\theta^2}{2} - i\frac{\theta^3}{6} + \frac{\theta^4}{24} + i\frac{\theta^5}{120} + \cdots\]

    Separating the real and imaginary terms:
    \[e^{(i\theta)} = \Bigl(1 - \frac{\theta^2}{2} + \frac{\theta^4}{24} +\cdots \Bigr) + i\Bigl(\theta  - \frac{\theta^3}{6}  + \frac{\theta^5}{120} + \cdots \Bigr)\]
    \[e^{(i\theta)} = \cos{(\theta)} + i\sin{(\theta)}\]

    Notice that this is the same as the polar form for a complex number, meaning that $z = x + iy$ can be written as $z=r(cos{(\theta)}+i\sin{(\theta)})$ \textbf{or} $z=e^{i\theta}$.

    \item
    Find the Taylor series for the function $f(x) = 2x e^{-6x}$ about $x = 1$

    $f'(x) = 2e^{-6x} - 12xe^{-6x}$

    $f''(x) = -12e^{-6x} - 12e^{-6x} + 72xe^{-6x} = -24e^{-6x} + 72xe^{-6x}$

    $f^{(3)}(x) = 144e^{-6x} + 72e^{-6x} - 432xe^{-6x} = 216e^{-6x} - 432xe^{-6x}$

    Substituting in $x=1$, get the following Taylor series:

    $f(x) = 2x e^{-6x} = 2e^{-6} + \Bigl(2e^{-6} - 12e^{-6}\Bigr)(x-1) + \Bigl(24e^{-6} + 72e^{-6}\Bigr)\frac{(x-1)^2}{2} + \Bigr(216e^{-6}-432e^{-6}\Bigr)\frac{(x-1)^3}{6}\dots$

    $f(x) = \frac{2}{e^6} - \frac{10(x-1)}{e^6} + \frac{96(x-1)^2}{2e^6} - \frac{216(x-1)^3}{6e^6} + \dots$

    $f(x) = \frac{2}{e^6} - \frac{10(x-1)}{e^6} + \frac{48(x-1)^2}{e^6} - \frac{36(x-1)^3}{e^6} + \dots$

    $f(x) = \frac{1}{e^6}\Bigl(2 - 10(x-1) + 48(x-1)^2 - 36(x-1)^3 + \dots\Bigr)$


\end{enumerate}




\end{document}