\documentclass[../main.tex]{subfiles}
\graphicspath{{\subfix{../images/}}}
\begin{document}
\section*{Term 1 Week 5}
\begin{enumerate}
    \item 
    \(\int (\frac{1-\sin{x}}{1+\sin{x}} \times \frac{1-\sin{x}}{1-\sin{x}}) dx\)\\

    \(\int \frac{1-2\sin{x}+\sin^2{x}}{1-\sin^2{x}} dx\)\\

    \(\int \frac{1-2\sin{x}+\sin^2{x}}{\cos^2{x}} dx\)\\

    \(\int \frac{1}{\cos^2{x}} dx-\int \frac{2\sin{x}}{\cos^2{x}}dx + \int \frac{\sin^2{x}}{\cos^2{x}} \)\\

    \(\int \sec^2{(x)}dx - \int \frac{2\sin{x}}{\cos{x}}\times \frac{1}{\cos{x}} dx + \int \tan^2{(x)}dx\)\\

    \(\int \sec^2{(x)}dx - \int 2\tan{(x)}\sec{(x)} dx + \int (\sec^2{(x)}-1) dx\)\\

    \(=\tan{(x)}-\sec{(x})+\tan{(x)}-x+c\)\\

    \(=2\tan{(x)}-\sec{(x})-x+c\)\\
    
    \item 
    Using Pythagoras:\\
    \(\sin^2{x}+\sin^2{2x}=\sin^2{3x}\)\\

    We need to rewrite each term so that they are all in terms of \(\sin{x}\)\\
    \(\sin^2{2x}=(\sin{2x})^2=(2\sin{x}\cos{x})^2=4\sin^2{x}\cos^2{x}\)\\
    \(=4\sin^2{x}(1-\sin^2{x})\)\\
    \(=4\sin^2{x}-4\sin^4{x}\)\\

    \(\sin{3x}=\sin{(2x+x)}=\sin{2x}\cos{x}+\cos{2x}\sin{x}\)\\
    \(=2\sin{x}\cos^2{x}+(1-2\sin^2{x})\sin{x}\)\\
    \(=2\sin{x}(1-\sin^2{x})+\sin{x}-2\sin^3{x}\)\\
    \(=2\sin{x}-2\sin^3{x}+\sin{x}-2\sin^3{x}\)\\
    \(=3\sin{x}-4\sin^3{x}\)\\
    
    Therefore, \(\sin^3{(3x)}=(3\sin{x}-4\sin^3{x})^3=16\sin^6{x}-24\sin^4{x}+9\sin^2{x}\)\\

    Substituting everything back into the original equation:\\
    \(\sin^2(x)+4\sin^2(x)-4\sin^4(x)=16\sin^6(x)-24\sin^4(x)+9\sin^2(x)\)\\
    \(16\sin^6(x)-20\sin^4(x)+4\sin^2(x)=0\)\\
    
    Simplifying and factorising:\\
    \(4\sin^6(x)-5\sin^4(x)+\sin^2(x)=0\)\\
    \(\sin^2(x)(4\sin^4(x)-5\sin^2(x)+1)=0\)\\
    \(\sin^2(x)(4\sin^2(x)-1)(\sin^2(x)-1)=0\)\\
    Difference of two squares:\\
    \(\sin^2(x)(2\sin(x)+1)(2\sin(x)-1)(\sin(x)+1)(\sin(x)-1)=0\)\\
    
    From here, we just examine each factor and see which gives valid solutions for our triangle.\\
    \(sin^2(x)=0\)\\
    \(x=0\) (Not valid)\\

    \(2\sin(x)+1=0\)\\
    \(\sin(x)=-\frac{1}{2}\) (Not valid)\\

    \(2\sin(x)-1=0\)\\
    \(\sin(x)=\frac{1}{2}\)\\
    \(x=\frac{\pi}{6}\) (valid)\\

    \(\sin(x)+1=0\)\\
    \(\sin(x)=-1\) (Not valid)\\

    \(\sin(x)-1=0\)\\
    \(\sin(x)=1\)\\
    \(x=\frac{\pi}{2}\) (Not valid as the side with \(\sin(2x)\) will have length of \(\sin(\pi)=0\))\\

    Therefore, the only valid solution is \(x=\frac{\pi}{6}\).\\
    
    \item 
    Separate the RHS into two factors:\\
    \(x^x=5^x \times 5^{25}\)\\

    Divide by \(5^x\):\\
    \(\frac{x^x}{5^x}=5^{25}\)\\

    Simplifying the LHS:\\
    \((\frac{x}{5})^x=5^{25}\)\\

    Take the fifth root:\\
    \((\frac{x}{5})^\frac{x}{5}=5^5\)\\

    Therefore, \(\frac{x}{5}=5\)\\
    \(x=25\)
    
\end{enumerate}

\end{document}