\documentclass[../main.tex]{subfiles}
\graphicspath{{\subfix{../images/}}}
\begin{document}
\section*{Term 2 Week 7}
\begin{enumerate}[itemsep=1cm]
    \item 
    Find all polynomials \(f(x)\) such that \(f(2x)=f'(x).f''(x)\)\\

    Start by supposing that the polyomial is of degree \textit{n}. Then comparing degrees on each side we have the following:\\
    \(x^n=x^{n-1} \times x^{n-2}=x^{2n-3}\)\\

    This means that \(n=2n-3\), giving \(n=3\), therefore \(f(x)\) is a cubic. Note that this assumes that \(n-1\) is non-zero.\\

    If \(f(x)\) was linear, meaning \(n-1=0\), then the degree on the right would be zero, the second derivative would be zero, giving \(f(x)=0\) as one valid solution.\\

    Looking at the cubic solution, we examine the coefficients:\\
    \(f(x)=ax^3+bx^2+cx+d\)\\

    \(f'(x)=3ax^2+2bx+c\)\\
    \(f''(x)=6ax+2b\)\\
    \(f(2x)=a(2x)^3+b(2x)^2+c(2x)+d=8ax^3+4bx^2+2cx+d\)\\

    Equating the two sides one term at a time:\\
    \(x^3\) terms:\\
    \(8ax^3=3ax^2 \times 6ax=18a^2x^3\)\\
    Therefore, \(8a=18a^2\), meaning \(a=\frac{4}{9}\)\\

    This makes our cubic \(f(x)=\frac{4}{9}x^3+bx^2+cx+d\)\\

    \(f'(x)=\frac{4}{3}x^2+2bx+c\)\\
    \(f''(x)=\frac{8}{3}x+2b\)\\
    \(f(2x)=\frac{32}{9}x^3+4bx^2+2cx+d\)\\

    \(x^2\) terms:\\
    \(4bx^2=\frac{4}{3}x^2 \times 2b + 2bx \times \frac{8}{3}x\)\\
    \(4b=\frac{8}{3}b+\frac{16}{3}b=8b\)\\
    \(4b=8b \Rightarrow b=0\)\\

    This makes our cubic \(f(x)=\frac{4}{9}x^3+cx+d\)\\

    \(f'(x)=\frac{4}{3}x^2+c\)\\
    \(f''(x)=\frac{24}{9}x\)\\
    \(f(2x)=\frac{32}{9}x^3+2cx+d\)\\

    \(x\) terms:\\
    \(2c=\frac{24}{9}c \Rightarrow c=0\)\\
    
    This makes our cubic \(f(x)=\frac{4}{9}x^3+d\)\\

    \(f'(x)=\frac{4}{3}x^2\)\\
    \(f''(x)=\frac{24}{9}x\)\\
    \(f(2x)=\frac{32}{9}x^3+d\)\\

    Constant term must therefore be zero.\\

    This means the only possible solutions for \(f(x)\) are \(f(x)=0\) and \(f(x)=\frac{4}{9}x^3\).\\
    
    \item 
    
    We know that the sum of the digits 1-9 is 45, which is a multiple of 3. Therefore, $X + Y + Z = 0 \mod{3}$.

    Since $X+Y=Z$, this means that $X+Y \mod{3}=-Z\mod{3}$. It follows that $X+Y \mod{3}=Z\mod{3}$.

    This also means that $2Z \mod{3}=0$.

    Since $Z$ is a power of a prime and also a multiple of 3, it must therefore be a power of 3. The only 3-digit multiples of 3 are 243 and 729. 243 is too small to be the sum of two other 3-digit numbers where we are using all of the digits from 1-9, therefore $Z$=729.

    Now that we know $Z$, we can work out $X$ and $Y$ by inspection. If we write $X=abc$ and $Y=def$, we know that $c+f=9$ (they can't add to 19).

    This means that $b+e=12$, as they can't possibly add to just 2. And since that means there is a carryover into the hundreds column, $a+d=6$.

    With $Z=729$, the only digits remaining are 1,3,4,5,6,8. There is only one way to get 12 as a sum of any two of those numbers, therefore since the digits of $X$ are greater than those of Y, $b=8$ and $e=4$. 

    This leaves the digits 1,3,5,6. There is only one option for the remaining values of $X$ and $Y$. $a=5, d=1$ and $c=6, f=3$.

    Therefore, our solution is:

    $X=586\\
    Y=143\\
    Z=729$

    \item
    \(e^{i(A-B)}=e^{iA}e^{-iB}\)\\
    This means that:\\
    $
    \!
    \begin{aligned}[t]
    \cos{(A-B)+i\sin{(A-B)}}
    &=(\cos{(A)+i\sin{(A)}})(\cos{(-B)+i\sin{(-B)}})\\
    &=(\cos{(A)+i\sin{(A)}})(\cos{(B)-i\sin{(B)}})
    \end{aligned}\\
    $
    
    Equating real and imaginary parts:\\
    $
    \!
    \begin{aligned}[t]
    \cos{(A-B)}
    &=\cos{(A)}\cos{(B)}+\sin{(A)}\sin{(B)}\\
    \sin(A-B)
    &=\cos{(B)}\sin{(A)}-\cos{(A)}\sin{(B)}\\
    \end{aligned}\\
    $

    Substituting \(-B\) for \(B\) in the second equation:\\
    \(\sin{(A+B)}=\sin{(A)}\cos{(-B)}-\cos{(A)}\sin{(-B)}=\sin{(A)}\cos{(B)}+\cos{(A)}\sin{(B)}\)\\

    \item 
    \(\int \sin^2{(x)}\cos^2{(x)} \,dx\)\\
    Use the Double Angle identities to rewrite each factor:\\
    \(\cos{2x}=2\cos^2{x}-1\\
    \cos^2{x}=\frac{1}{2}(1+\cos{2x})\)\\
    \(\cos{2x}=1-2\sin^2{x}\\
    \sin^2{x}=\frac{1}{2}(1-\cos{2x})\)\\

    Substituting into the integral:\\
    \(\int \frac{1}{2}(1+\cos{2x}) \times \frac{1}{2}(1-\cos{2x}) \,dx\)\\
    \(\frac{1}{4}\int (1+\cos{2x})(1-\cos{2x}) \,\)\\
    \(\frac{1}{4}\int (1-\cos^2{2x}) \,dx\)\\

    Use the Double Angle identity a second time:\\
    \(\cos{4x}=2\cos^2{2x}-1\)\\
    \(\cos^2{2x}=\frac{1}{2}(1+\cos{4x})\)\\

    Substitute into the integral:\\
    \(\frac{1}{4}\int (1-\frac{1}{2}(1+\cos{4x}))\,dx\)\\
    \(\frac{1}{4}\int (1-\frac{1}{2}-\frac{1}{2}\cos{4x})\,dx\)\\
    \(\frac{1}{4}\int (\frac{1}{2}-\frac{1}{2}\cos{4x})\,dx\)\\
    \(\frac{1}{4}\int \frac{1}{2}(1-\cos{4x})\,dx\)\\
    \(\frac{1}{8}\int (1-\cos{4x})\,dx\)\\

    Finally, integrate term by term:\\
    \(\frac{1}{8}\int (1-\cos{4x})\,dx=\frac{1}{8}(x-\frac{\sin{4x}}{4})+c=\frac{x}{8}-\frac{\sin{4x}}{32}+c\)\\
    
    \end{enumerate}

\end{document}