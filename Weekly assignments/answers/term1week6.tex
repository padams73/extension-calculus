\documentclass[../main.tex]{subfiles}
\graphicspath{{\subfix{../images/}}}
\begin{document}
\section*{Term 1 Week 6}
\begin{enumerate}
    \item 
    \((\ln{x})^2+(\ln{2x})^2=(\ln{3x})^2\)\\
    \((\ln{x})^2+(\ln{(x)}+\ln{2})^2=(\ln{(x)}+\ln{3})^2\)\\
    \((\ln{x})^2+(\ln{x})^2+2\ln{x}\ln{2}+(\ln{2})^2=(\ln{x})^2+2\ln{x}\ln{3}+(\ln{3})^2\)\\
    \((\ln{x})^2+2(\ln{2}-\ln{3})\ln{x}+(\ln{2})^2-(\ln{3})^2\)=0\\
    \((\ln{x})^2+2(\ln{2}-\ln{3})\ln{x}+(\ln{2}+\ln{3})(\ln{2}-\ln{3})\)=0\\
    \((\ln{x})^2+2\ln{\frac{2}{3}}\ln{x}+\ln{6}\ln{\frac{2}{3}}=0\)\\

    Solving for \(\ln{x}\) using the Quadratic Formula:\\
    \(\ln{x}=\frac{-2\ln{\frac{2}{3}}\pm \sqrt{(-2\ln{\frac{2}{3}})^2-4\ln{6}\ln{\frac{2}{3}}}}{2}\)\\

    \(\ln{x}=\frac{-2\ln{\frac{2}{3}}\pm \sqrt{4(\ln{\frac{2}{3}})^2-4\ln{6}\ln{\frac{2}{3}}}}{2}\)\\

    Note: \(-2\ln{\frac{2}{3}}=\ln{(\frac{2}{3})^2}=\ln{(\frac{3}{2})^2=2\ln{\frac{3}{2}}}\)\\

    \(\ln{x}=\frac{2\ln{\frac{3}{2}}\pm 2\sqrt{(\ln{\frac{2}{3}})^2-\ln{6}\ln{\frac{2}{3}}}}{2}\)\\

    \(\ln{x}=\ln{\frac{3}{2}}\pm \sqrt{(\ln{\frac{2}{3}})^2-\ln{6}\ln{\frac{2}{3}}}\)\\

    \(\ln{x}=\ln{\frac{3}{2}}\pm \sqrt{(\ln{\frac{2}{3}})(\ln{\frac{2}{3}}-\ln{6})}\)\\

    \(\ln{x}=\ln{\frac{3}{2}}\pm \sqrt{\ln{\frac{2}{3}}\ln{\frac{1}{9}}}\)\\

    Solutions:\\
    $
    \begin{aligned}[c]
    \ln{x}=\ln{\frac{3}{2}}+ \sqrt{\ln{\frac{2}{3}}\ln{\frac{1}{9}}}\\
    x=\frac{3}{2}e^{\sqrt{\ln{\frac{2}{3}}\ln{\frac{1}{9}}}}\\
    x=3.85 \text{(2 dp)}\\
    \end{aligned}
    \begin{aligned}[c]
    \ln{x}=\ln{\frac{3}{2}}- \sqrt{\ln{\frac{2}{3}}\ln{\frac{1}{9}}}\\
    x=\frac{3}{2}\div e^{\sqrt{\ln{\frac{2}{3}}\ln{\frac{1}{9}}}}\\
    x=0.58 \text{(2 dp)}\\
    \end{aligned}
    $
    
    0.58 gives a negative side length, therefore x=3.85.\\
    
    \item 
    Use the change of base formula:\\
    \(\frac{\log{x}}{\log{5}}+\frac{\log{x}}{\log{7}}=\frac{\log{x}}{\log{25}}\)\\

    Rearrange so that it is equal to zero:\\
    \(\frac{\log{x}}{\log{5}}+\frac{\log{x}}{\log{7}}-\frac{\log{x}}{\log{25}}=0\)\\

    Factorise out the \(\log{x}\):\\
    \(\log{x}(\frac{1}{\log{5}}+\frac{1}{\log{7}}-\frac{1}{\log{25}})=0\)\\

    Since \(\frac{1}{\log{5}}+\frac{1}{\log{7}}-\frac{1}{\log{25}}\) can never be zero, \(\log{x}=0\).\\

    Therefore, \(x=1\).\\

    \item 
    \(4(x^2-4hx)-y^2+2hy+15h^2-4a^2=0\)\\

    Rearrange:\\
    \(4(x^2-4hx)-(y^2-2hy)+15h^2-4a^2=0\)\\

    Complete the square:\\
    \(4(x-2h)^2-16h^2-((y-h)^2-h^2)+15h^2-4a^2=0 \)\\
    \(4(x-2h)^2-16h^2-(y-h)^2+h^2+15h^2-4a^2=0 \)\\

    Rearrange:\\
    \(4(x-2h)^2-(y-h)^2=4a^2\)\\

    Divide by sides so it is equal to 1:\\
    \(\frac{(x-2h)^2}{a^2}-\frac{(y-h)^2}{4a^2}=1\)\\
    Therefore, it is a hyperbola.\\

    Tangent gradient:\\
    Implicitly differentiate:\\
    \(\frac{2(x-2h)}{a^2}-\frac{2(y-h)}{4a^2}\frac{dy}{dx}=0\)\\

    \(\frac{8(x-2h)}{4a^2}=\frac{2(y-h)}{4a^2}\frac{dy}{dx}\)\\

    \(8(x-2h)=2(y-h)\frac{dy}{dx}\)\\

    \(\frac{dy}{dx}=\frac{8(x-2h)}{2(y-h)}\)\\

    \(\frac{dy}{dx}=\frac{4x-8h}{y-h}\)\\

    At point \((p,q)\) the gradient is \(e^2 -1\).\\
    \(\frac{4p-8h}{q-h}=e^2-1\)\\

    Since \(e^2=1+\frac{b^2}{a^2}\):\\
    \(\frac{4p-8h}{q-h}=\frac{b^2}{a^2}\)\\
    
    From the hyperbola equation, \(a^2=a^2\) and \(b^2=4a^2\).\\
    Substituting in:\\
    \(\frac{4p-8h}{q-h}=\frac{4a^2}{a^2}\)\\

    \(\frac{4p-8h}{q-h}=4\)\\

    \(4p-8h=4q-4h\)\\
    \(4h=4p-4q\)\\
    \(h=p-q\)\\
    
\end{enumerate}

\end{document}